\documentclass[a4paper]{article}
\usepackage[colorlinks=true,allcolors=blue]{hyperref}
\usepackage{amsmath}
\usepackage{graphicx}
\usepackage{amsfonts}
\usepackage[none]{hyphenat} %this is not necessary for the students to figure out, but a good thing to know
% it also sometimes gets rid of the overfull/underfull box warning



\title{Formulas exercise}
\author{Me, myself and I}
\date{\today}

\begin{document}


\maketitle
	


\section*{Formulas and mathematical notation}
\begin{equation*}
    E = mc^2
\end{equation*}


\begin{align*} 
2x - 6 &=  8 \\ 
2x &=  8 + 6\\
  x  &= \frac{8 + 6}{2}\\
  x &= 7
\end{align*}

\begin{center}
\noindent 3H\textsubscript{2} + N\textsubscript{2} $\rightarrow$ 2NH\textsubscript{3}    
\end{center}


\begin{equation} \label{eq:resistance}
    R = \frac{L}{{\sigma A}} = \frac{{\rho L}}{A}
\end{equation}	

\noindent Equation \ref{eq:resistance} describes electrical resistance. The symbol $\sigma$ is called sigma, $\rho$ is called rho.\\

\noindent $\mathbb{N}$ denotes the set of natural numbers. Therefore, $ \mathbb{N}=\{1,2,3,\ldots,\infty\} $

\newpage



		
\end{document}

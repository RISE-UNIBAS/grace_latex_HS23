\documentclass[10pt,a4paper]{article}
\usepackage{allrunes} % for runes
\usepackage[safe]{tipa} % for superscribed letters using \sups{}{}
\renewcommand{\sups}[2]{\textipa{\tipaUpperaccent[.2ex]{%
 			\lower.8ex\hbox{\super{#2}}}{#1}}} % redefines appearance of superscribed letter
\usepackage[icelandic]{babel} 
\usepackage[T1]{fontenc} % specifies font encoding; T1 includes accented characters as individual glyphs, summing up to 256 characters (as opposed to the default font encoding (OT1) of TeX, which is 7-bit and uses fonts that have 128 glyphs)
\usepackage[utf8]{inputenc} % for UTF 8 support
\usepackage{yfonts} % includes gothic, fraktur and schwabacher font
\usepackage{url} % let's urls appear in typewriter style
\title{Special characters and fonts}
\date{\vspace{-5ex}} % removes the date in an easy, but a bit unelegant way

\begin{document}

\maketitle

\section{Package \texttt{allrunes}}
\url{https://ctan.math.illinois.edu/fonts/allrunes/allrunes.pdf}\par\medskip

\subsection{Germanic Runes}
\textarc{fu\th \H j Rs M d}

\subsection{Anglo-Frisian Runes}
\textara{fu\th \H j Rs M d}

\subsection{Normal Runes}
\textarn{fu\th \H j Rs M d}

\subsection{Medieval Runes}
\textarm{fu\th \H j Rs M d}

\section{Middle High/Low German(ic) characters}

\subsection{Superscripts}
z\r{u} mînem br\r{u}der\\
r\sups{o}{e}msche k\"unge

\subsection{Special characters 1}
þÞ æ Æ ð Ð

\subsection{Special characters 2}
Either find it in the symbols list (\url{https://www.cl.uni-heidelberg.de/courses/ss19/wissschreib/material/symbols-a4.pdf}) or use the \texttt{xunicode} package (and another compiler, XeLaTeX).

\section{Special fonts}
A lot of special fonts are available online -- you can even produce documents using Comic Sans (with a bit of passion and installation), or \textgoth{Gothic script}. 

\end{document}